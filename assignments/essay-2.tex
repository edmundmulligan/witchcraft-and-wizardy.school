\documentclass[a4paper, 12pt]{article}

% include required packages
\usepackage[a4paper, margin=2cm]{geometry} % Set page margins
\usepackage[round,authoryear]{natbib}      % For citations
\usepackage{url}                           % For formatting URLs
\usepackage{indentfirst}                   % Indent first paragraph of sections
\usepackage{booktabs}                      % For better table formatting
\usepackage{float}                         % For [H] table placement
\usepackage{enumitem}                      % For customizing lists
\usepackage{tikz}                          % For drawing diagrams
\usetikzlibrary{positioning}               % For relative positioning

% set paragraph and list formatting
\setlength{\parindent}{1.5em}
\setlength{\parskip}{0pt}
\setlist[itemize]{noitemsep,topsep=0pt,parsep=0pt,partopsep=0pt,after=\vspace{0.5em}}


% set title information
\title{A critical reflection on refactoring the Web Witchcraft and Wizardry project}
\author{Edmund Mulligan}
\date{\today}

% document start
\begin{document}

% set bibliography style and page numbering
\bibliographystyle{plainnat}
\pagenumbering{arabic}

\maketitle

\begin{abstract}
\textit{Abstract goes here.}
\end{abstract}

\section*{Introduction}

Web Witchcraft and Wizardry \citep{mulligan2024witchcraftandwizardry} is a website for children  
to learn the basics of web development through fantasy-themed metaphors. The original version of 
the website was neither accessible (except by accident) nor responsive. This essay reflects on the 
design and implementation of a refactored version of the project, focusing on the accessibility, 
usability, and quality of the new code.

information arcitecture and layout choices
navigation design
responsive design techniques
accessibility features implemented


grid vs flexbox
use of media queries
semantic HTML
ARIA roles and attributes
colour contrast
keyboard navigation
alt text for images
form accessibility

performance improvements (minification, image optimisation)

dry code \citep{hunt1999pragmatic}

automated testing (Lighthouse, axe, pa11y, Wave)
manual testing (screen readers, keyboard only navigation)

\section*{Conclusion}

% Include word count N.B. Run wordcount.sh to generate wordcount.txt before compiling
\section*{Word count}
Word count: \input{wordcount.txt} words (excluding references)
\\
\bibliography{EdmundLibrary}

\end{document}
