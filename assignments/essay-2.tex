\documentclass[a4paper, 12pt]{article}

% include required packages
\usepackage[a4paper, margin=2cm]{geometry} % Set page margins
\usepackage[round,authoryear]{natbib}      % For citations
\usepackage{url}                           % For formatting URLs
\usepackage{indentfirst}                   % Indent first paragraph of sections
\usepackage{booktabs}                      % For better table formatting
\usepackage{float}                         % For [H] table placement
\usepackage{enumitem}                      % For customizing lists
\usepackage{tikz}                          % For drawing diagrams
\usetikzlibrary{positioning}               % For relative positioning

% set paragraph and list formatting
\setlength{\parindent}{1.5em}
\setlength{\parskip}{0pt}
\setlist[itemize]{noitemsep,topsep=0pt,parsep=0pt,partopsep=0pt,after=\vspace{0.5em}}


% set title information
\title{A critical reflection on refactoring the Web Witchcraft and Wizardry project}
\author{Edmund Mulligan}
\date{\today}

% document start
\begin{document}

% set bibliography style and page numbering
\bibliographystyle{plainnat}
\pagenumbering{arabic}

\maketitle

\begin{abstract}
\textit{Abstract goes here.}
\end{abstract}

\section*{Introduction}

Web Witchcraft and Wizardry \citep{mulligan2024witchcraftandwizardry} is a website for children  
to learn the basics of web development through fantasy-themed metaphors. The original version of 
the website was neither accessible (except by accident) nor responsive. This essay reflects on the 
design and implementation of a refactored version of the project, focusing on accessibility, 
usability, and quality of the new code. Detailed decisions are documented as comments in 
the source code, which should be read alongside this essay, so the essay focuses on higher level
design decisions and reflections.

The development process used was iterative and incremental \citep{beck2004extreme}, and 
while there are many criticisms of this approach, particularly around its lack of documentation, 
endles changes and lack of scalability to large teams (see, for example, \citep{boehm2004balancing}), 
it was well suited to this small project with a single developer fulfilling all team roles.

\section*{Responsiveness and Accessibility}
A primary goal of the refactor was to make the website responsive and accessible.

\section*{Site Architecture and Navigation}

information arcitecture and layout choices
navigation design
site navigation vs page navigation

\section*{Third party resources}


\section*{Testing}

\section*{Challenges and Solutions}

responsive design techniques
Fluid grids, flexible images, and media queries \citep{marcotte2010responsive}
prefer responsive 
Viewport cutoffs
< 80px too small - unusable
80px - 200px very small - give warning but still usable
200px - 800px small - mobile layout - prefer vertical stacking
> 800phx - desktop layout - prefer horizontal layout


accessibility features implemented


grid vs flexbox
use of media queries
semantic HTML
ARIA roles and attributes
colour contrast
keyboard navigation
alt text for images
form accessibility

performance improvements (minification, image optimisation)

dry code \citep{hunt1999pragmatic}

automated testing (Lighthouse, axe, pa11y, Wave)
manual testing (screen readers, keyboard only navigation)

\section*{Conclusion}

% Include word count N.B. Run wordcount.sh to generate wordcount.txt before compiling
\section*{Word count}
Word count: \input{wordcount.txt} words (excluding references)
\\
\bibliography{EdmundLibrary}

\end{document}
